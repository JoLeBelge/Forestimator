\documentclass[12pt,a4paper]{report}

\usepackage[utf8]{inputenc}
\usepackage[T1]{fontenc}
\usepackage[french]{babel}
\usepackage{graphicx}
\usepackage{hyperref}
\usepackage{tcolorbox}
\usepackage{geometry}
\usepackage{setspace}
\usepackage{titlesec}

\geometry{margin=2.5cm}
\setstretch{1.2}

% Styles de boîtes
\tcbset{
  colback=gray!5,
  colframe=gray!80!black,
  fonttitle=\bfseries,
  coltitle=black,
  boxrule=0.5pt,
  arc=3pt,
  left=6pt,
  right=6pt,
  top=4pt,
  bottom=4pt
}

\newtcolorbox{procedurebox}[1]{title={Procédure : #1}}
\newtcolorbox{warningbox}{colback=red!5,colframe=red!60!black,title={Attention}}

\begin{document}

% -------------------------------------------------------------------
% PAGE DE TITRE
% -------------------------------------------------------------------
\begin{titlepage}
\centering
{\Huge \textbf{Manuel d’utilisation de l’application Forestimator}}\\[1.5cm]

{\Large Rédigé par Thierry Thissen, Jonathan Lisein et MistralAi}\\[1cm]

\includegraphics[width=0.25\textwidth]{LogoUliegeForest.jpg}\hspace{2cm}
\includegraphics[width=0.25\textwidth]{LogoForestimator.png}\\[2cm]

{\large Université de Liège – Gembloux Agro-Bio Tech}\\[0.5cm]
{\large Date de rédaction : \today}\\

\vfill
\end{titlepage}

% -------------------------------------------------------------------
% TABLE DES MATIÈRES
% -------------------------------------------------------------------
\tableofcontents
\newpage

% -------------------------------------------------------------------
\chapter{Introduction}

Forestimator est un outil développé pour faciliter l’analyse et la visualisation de données forestières.  
Il existe deux formes du projet : la \textbf{page web Forestimator} et l’\textbf{application mobile Forestimator}.  
Ce manuel est consacré exclusivement à l’application mobile.

Pour plus d’informations générales, consultez :  
\url{https://forestimator.gembloux.ulg.ac.be/documentation/forestimator/}.

% -------------------------------------------------------------------
\chapter{Installation}

\section*{Avant l’installation}
L’application \textbf{Forestimator Mobile} est disponible pour Android et iOS.  
Elle peut être installée en Belgique, aux Pays-Bas, en France, en Allemagne et au Luxembourg.

\begin{procedurebox}{Installation de Forestimator}
\begin{enumerate}
    \item Ouvrez le \textbf{Google Play Store} ou l’\textbf{Apple App Store}.
    \item Recherchez le terme \emph{Forestimator}.
    \item Appuyez sur \textbf{Installer}.
    \item Attendez la fin de l’installation.
    \item Appuyez sur \textbf{Ouvrir} pour lancer Forestimator pour la première fois.
\end{enumerate}
\end{procedurebox}

% -------------------------------------------------------------------
\chapter{Première utilisation}

Après avoir ouvert l’application, vous verrez l’écran principal avec la couche IGN en arrière-plan.

\includegraphics[width=0.25\textwidth]{StartScreen.png}

À un certain moment, l’application vous demandera certaines autorisations.  
Ces autorisations ne sont pas indispensables pour le fonctionnement global, mais certaines fonctions avancées nécessitent qu’elles soient accordées.

\includegraphics[width=0.25\textwidth]{Permissions.png}

% -------------------------------------------------------------------
\chapter{L’écran principal}

\includegraphics[width=0.25\textwidth]{StartScreenDescription.png}

La partie inférieure de l’écran présente une \textbf{barre de menu} composée de trois icônes :

\begin{itemize}
    \item \includegraphics[width=0.03\textwidth]{icons/tree.png} : ouvre le menu des outils.  
    Il contient des fonctions pour rechercher un lieu en Wallonie ou pour localiser votre position GPS (si autorisée).
    \item \includegraphics[width=0.03\textwidth]{icons/hexagon.png} : ouvre le menu des polygones, où vous pouvez créer des formes sur la carte et demander une analyse de surface sur nos serveurs.
    \item \includegraphics[width=0.03\textwidth]{icons/eye.png} : ouvre le sélecteur de couches.  
    Il permet de changer la carte, la visibilité des polygones et marqueurs, de consulter le catalogue ou de télécharger des cartes pour une utilisation hors ligne.
\end{itemize}

En haut à gauche : \includegraphics[width=0.03\textwidth]{icons/settings.png} pour accéder aux paramètres généraux.  
Au centre en haut : un bouton affichant \textbf{« Mode en ligne »} permet de basculer entre le mode en ligne et le mode hors ligne.

\begin{warningbox}
La carte peut afficher jusqu’à trois couches simultanément.  
Certaines couches ont une échelle de zoom limitée : elles disparaîtront si vous zoomez trop loin.  
Un symbole d’avertissement apparaît alors en haut à droite.
\end{warningbox}

% -------------------------------------------------------------------
\chapter{Paramètres}

\includegraphics[width=0.25\textwidth]{Settings.png}

Le menu des paramètres comprend quatre sous-menus :
\begin{itemize}
    \item \textbf{Permissions} : visualiser et modifier les autorisations accordées.
    \item \textbf{À propos} : affiche la version et les informations de financement.
    \item \textbf{Contact} : lien vers le site principal et adresses e-mail pour les retours ou bugs.
    \item \textbf{Confidentialité}.
\end{itemize}

% -------------------------------------------------------------------
\chapter{Le menu des outils}

\includegraphics[width=0.25\textwidth]{ToolMenu2.png}

Le menu des outils s’ouvre en appuyant sur l’icône de l’arbre.  
Deux outils principaux apparaissent :

\begin{itemize}
    \item L’icône GPS indique si le mobile connaît votre position.  
    Grise = non localisé, rouge = position trouvée. En appuyant dessus, la carte se centre sur votre position.
    \item L’icône de loupe permet de rechercher un lieu en Wallonie via le serveur \emph{OpenStreetMap Nominatim}.
\end{itemize}

\includegraphics[width=0.25\textwidth]{ToolMenu3.png}

% -------------------------------------------------------------------
\chapter{Le menu des polygones}

\includegraphics[width=0.25\textwidth]{PolygonMenu.png}

Le menu s’ouvre en appuyant sur l’icône de l’hexagone.  
Un encart en haut de l’écran indique « + Tapez ici pour ajouter un polygone ».

\begin{procedurebox}{Créer un nouveau polygone}
\begin{enumerate}
    \item Appuyez sur le bouton \textbf{+}.
    \item Donnez un nom court au polygone et validez.
    \item Une carte du polygone s’affiche avec ses informations : surface, périmètre, couleur, etc.
    \item Utilisez la corbeille pour supprimer, la palette pour changer la couleur, et l’icône d’analyse pour lancer une analyse.
\end{enumerate}
\end{procedurebox}

\includegraphics[width=0.25\textwidth]{NewPoly1.png}

\begin{procedurebox}{Ajouter des sommets}
\begin{enumerate}
    \item Appuyez sur \includegraphics[width=0.03\textwidth]{icons/add_circle.png}.
    \item Tapez sur la carte pour ajouter des points.
    \item Après 3 points, un triangle apparaît.  
          Vous pouvez diviser les arêtes en ajoutant de nouveaux sommets.
\end{enumerate}
\end{procedurebox}

\includegraphics[width=0.25\textwidth]{NewPoly4.png}

\begin{procedurebox}{Déplacer ou supprimer un sommet}
\begin{enumerate}
    \item Touchez un sommet : de nouvelles icônes apparaissent.  
    \item \includegraphics[width=0.03\textwidth]{icons/remove_circle.png} pour supprimer.  
    \item \includegraphics[width=0.03\textwidth]{icons/open_with.png} pour déplacer.  
    \item Une fois le déplacement terminé, appuyez à nouveau pour valider.
\end{enumerate}
\end{procedurebox}

\includegraphics[width=0.25\textwidth]{NewPoly5.png}

% -------------------------------------------------------------------
\chapter{Le sélecteur de couches}

L’icône \includegraphics[width=0.03\textwidth]{icons/eye.png} ouvre le sélecteur de couches.

\begin{itemize}
    \item Partie supérieure : contrôle de la visibilité des marqueurs et polygones.
    \item Partie centrale : trois cartes superposables avec transparence ajustable.
    \item Partie inférieure : accès aux catalogues (\includegraphics[width=0.03\textwidth]{icons/layers.png} pour le catalogue en ligne, \includegraphics[width=0.03\textwidth]{icons/download_for_offline.png} pour les couches téléchargées).
\end{itemize}

\begin{procedurebox}{Sélectionner une couche}
\begin{enumerate}
    \item Appuyez sur le texte « Appuyez ici pour ajouter une couche du catalogue ».
    \item Ouvrez le catalogue (\includegraphics[width=0.03\textwidth]{icons/layers.png}).
    \item Sélectionnez une couche : un cercle vert numéroté indique la position (1–3).
\end{enumerate}
\end{procedurebox}

\begin{procedurebox}{Supprimer une couche}
\begin{enumerate}
    \item Ouvrez le catalogue depuis la tuile correspondante.
    \item Appuyez sur le cercle vert pour désélectionner la carte.
\end{enumerate}
\end{procedurebox}

% -------------------------------------------------------------------
\chapter{Le catalogue des couches}

\includegraphics[width=0.25\textwidth]{catalogue.png}

Le catalogue présente toutes les couches disponibles depuis le serveur, regroupées en catégories :
\begin{itemize}
\item Conditions stationnelles
\item Guide des stations
\item Cartes de référence
\item Cartographie des peuplements forestiers
\item État sanitaire de la pessière
\item Adéquation des essences aux conditions stationnelles
\item Couches thématiques (catalogue de stations)
\end{itemize}

\includegraphics[width=0.25\textwidth]{catalogueLayerTile.png}

Chaque couche est décrite par des icônes :
\begin{itemize}
    \item \includegraphics[width=0.03\textwidth]{icons/download.png} : couche téléchargeable.
    \item \includegraphics[width=0.03\textwidth]{icons/disk.png} : déjà téléchargée.
    \item \includegraphics[width=0.03\textwidth]{icons/legend_toggle.png} : légende disponible.
    \item \includegraphics[width=0.03\textwidth]{icons/picture_as_pdfBrown.png} : documentation disponible.
    \item \includegraphics[width=0.03\textwidth]{icons/pentagram.png} : sélectionnée pour l’analyse surfacique.
    \item \includegraphics[width=0.03\textwidth]{icons/locationPin.png} : sélectionnée pour l’analyse ponctuelle.
\end{itemize}

% -------------------------------------------------------------------
\chapter{Analyses}

Forestimator Mobile permet deux types d’analyses : ponctuelles et surfaciques.

\begin{procedurebox}{Analyse ponctuelle}
\begin{enumerate}
    \item Tapez sur la carte à l’endroit voulu.
    \item Une ligne de prévisualisation s’affiche avec la légende et la couleur.
    \item Tapez de nouveau pour obtenir une analyse complète des couches sélectionnées.
    \item Un appui long permet d’accéder directement à l’analyse complète.
\end{enumerate}
\end{procedurebox}

\includegraphics[width=0.25\textwidth]{anaPreview.png}

\begin{warningbox}
L’analyse ponctuelle ne s’effectue que sur la première couche du sélecteur.  
Certaines cartes (IGN, orthophotos) ne permettent pas de prévisualisation.
\end{warningbox}

\begin{procedurebox}{Analyse de surface}
\begin{enumerate}
    \item Appuyez sur l’icône de l’hexagone.
    \item Ouvrez la liste des polygones.
    \item Sélectionnez le polygone voulu.
    \item Appuyez sur \includegraphics[width=0.03\textwidth]{icons/analytics.png} pour lancer l’analyse.
\end{enumerate}
\end{procedurebox}

\includegraphics[width=0.25\textwidth]{anaPt.png}

\begin{procedurebox}{Sélectionner des couches pour l’analyse}
\begin{enumerate}
    \item Ouvrez le sélecteur de couches.
    \item Appuyez sur \includegraphics[width=0.03\textwidth]{icons/layers.png} pour ouvrir le catalogue.
    \item Choisissez une couche, puis :
        \begin{itemize}
        \item \includegraphics[width=0.03\textwidth]{icons/pentagram.png} pour analyse surfacique.
        \item \includegraphics[width=0.03\textwidth]{icons/locationPin.png} pour analyse ponctuelle.
        \end{itemize}
\end{enumerate}
\end{procedurebox}

% -------------------------------------------------------------------
\chapter{Mode en ligne / hors ligne}

Le mode \textbf{hors ligne} permet d’utiliser Forestimator sans connexion, par exemple en forêt.

\section*{Avant l’utilisation hors ligne}
\begin{itemize}
    \item Téléchargez au moins une couche pour l’utiliser hors connexion.
    \item L’affichage hors ligne supporte une seule couche à la fois.
    \item L’analyse surfacique n’est pas disponible, mais vous pouvez tracer des polygones.
    \item La localisation GPS fonctionne, mais la recherche de lieux est désactivée.
\end{itemize}

\begin{procedurebox}{Basculer en mode hors ligne}
\begin{enumerate}
    \item Appuyez sur le bouton « Forestimator en ligne ».
    \item Le texte devient « Forestimator terrain » et la carte est chargée depuis la mémoire.
\end{enumerate}
\end{procedurebox}

\includegraphics[width=0.25\textwidth]{offlineMode.png}

\begin{procedurebox}{Revenir en mode en ligne}
\begin{enumerate}
    \item Appuyez sur « Forestimator terrain ».
\end{enumerate}
\end{procedurebox}

\begin{warningbox}
Les couches téléchargées peuvent être combinées avec des couches en ligne.  
Une petite icône dans le sélecteur indique si une couche provient du disque ou du serveur.
\end{warningbox}

% -------------------------------------------------------------------
\chapter*{Références}
\addcontentsline{toc}{chapter}{Références}

Les références bibliographiques et documentations techniques seront ajoutées ici ultérieurement.

% Exemple d'appel BibTeX :
% \bibliographystyle{plain}
% \bibliography{references}

\end{document}
