\documentclass[12pt,a4paper]{report}
\usepackage[utf8]{inputenc}
\usepackage[french]{babel}
\usepackage{graphicx}
\usepackage{hyperref}
\usepackage{float}
\usepackage{caption}
\usepackage{enumitem}
\usepackage{tcolorbox}
\usepackage{geometry}

\geometry{a4paper, left=25mm, right=25mm, top=25mm, bottom=25mm}
\title{Manuel d'utilisation de l'application Forestimator}
\author{}
\date{\today}

\tcbuselibrary{skins,breakable}
\newtcolorbox{instructionbox}[1]{
    colback=white,
    colframe=orange!75!black,
    fonttitle=\bfseries,
    colbacktitle=orange!75!black,
    coltitle=white,
    title=#1,
    breakable,
    enhanced,
    attach boxed title to top left={xshift=5mm, yshift=-3mm},
}

\newtcolorbox{warningbox}[1]{
    colback=white,
    colframe=red!75!black,
    fonttitle=\bfseries,
    colbacktitle=red!75!black,
    coltitle=white,
    title=#1,
    breakable,
    enhanced,
    attach boxed title to top left={xshift=5mm, yshift=-3mm},
}

\begin{document}

\maketitle
\tableofcontents

\chapter{Introduction}
Forestimator est un outil cartographique conçu pour la gestion des forêts en Wallonie. Il permet aux utilisateurs de visualiser des cartes, d'analyser des surfaces et des points spécifiques, et d'accéder à des informations géographiques détaillées. Pour plus d'informations, consultez la documentation officielle : \url{https://forestimator.gembloux.ulg.ac.be/documentation/forestimator/}.

\section{À propos de Forestimator}
Forestimator offre des outils cartographiques pour la gestion des forêts wallonnes. Il est accessible via une application mobile disponible sur Android et iOS.

\chapter{Installation}

\section{Avant l'installation}
\begin{itemize}
    \item Forestimator Mobile est disponible pour Android et iOS.
    \item L'application peut être installée en Belgique, aux Pays-Bas, en France, en Allemagne et au Luxembourg.
\end{itemize}

\section{Procédure d'installation}
\begin{instructionbox}{Procédure d'installation}
\begin{enumerate}
    \item Ouvrez le Google Play Store ou l'App Store d'Apple.
    \item Recherchez le terme \og forestimator \fg.
    \item Appuyez sur \og Installer \fg.
    \item Attendez la fin de l'installation.
    \item Appuyez sur \og Ouvrir \fg{} pour lancer Forestimator pour la première fois.
\end{enumerate}
\end{instructionbox}

\chapter{Première utilisation}

\section{Écran principal}
Lors de la première ouverture de l'application, vous verrez l'écran principal avec la couche IGN en arrière-plan.

\begin{figure}[H]
    \centering
    \includegraphics[width=0.25\textwidth]{StartScreen.png}
    \caption{Écran principal de Forestimator.}
\end{figure}

\section{Permissions}
À un certain moment, l'application demandera des permissions. Celles-ci ne sont jamais nécessaires pour le fonctionnement de l'application, mais certaines tâches spécifiques ne peuvent être réalisées sans elles.

\begin{figure}[H]
    \centering
    \includegraphics[width=0.25\textwidth]{Permissions.png}
    \caption{Demande de permissions.}
\end{figure}

\chapter{L'écran principal}

\section{Description}
L'écran principal de Forestimator contient plusieurs éléments interactifs.

\begin{figure}[H]
    \centering
    \includegraphics[width=0.25\textwidth]{StartScreenDescription.png}
    \caption{Description de l'écran principal.}
\end{figure}

\section{Barre de menu inférieure}
En bas de l'écran, vous trouverez une barre de menu avec trois éléments :
\begin{itemize}
    \item \textbf{À gauche :} \includegraphics[width=0.05\textwidth]{icons/tree.png} (icône d'arbre) pour ouvrir le menu des outils.
    \item \textbf{Au centre :} \includegraphics[width=0.05\textwidth]{icons/hexagon.png} (icône d'hexagone) pour ouvrir le menu des polygones.
    \item \textbf{À droite :} \includegraphics[width=0.05\textwidth]{icons/eye.png} (icône d'œil) pour ouvrir le sélecteur de couches.
\end{itemize}

\section{Barre de menu supérieure}
En haut de l'écran, vous trouverez :
\begin{itemize}
    \item \textbf{À gauche :} \includegraphics[width=0.05\textwidth]{icons/settings.png} (icône de paramètres) pour accéder aux réglages globaux ou aux informations sur l'application.
    \item \textbf{Au centre :} Un bouton indiquant \og Mode en ligne \fg{} pour basculer entre les modes en ligne et hors ligne.
\end{itemize}

\begin{warningbox}{Attention}
La carte peut afficher jusqu'à trois couches en même temps. Le zoom sur certaines couches est restreint, ce qui les fait disparaître si vous zoomez trop loin. Cela est indiqué par un signe d'avertissement en haut à droite de l'écran lorsque cela se produit.
\end{warningbox}

\chapter{Paramètres}

\section{Description}
Dans les paramètres, vous pouvez modifier les permissions accordées ou trouver des informations sur l'application et ses créateurs.

\begin{figure}[H]
    \centering
    \includegraphics[width=0.25\textwidth]{Settings.png}
    \caption{Menu des paramètres.}
\end{figure}

\section{Sous-menus}
\begin{itemize}
    \item \textbf{Permissions :} Consultez et modifiez les permissions accordées.
    \item \textbf{À propos :} Indique la version de Forestimator et les informations sur le financement.
    \item \textbf{Contact :} Adresse du site web principal et emails pour poser des questions ou signaler des bugs.
    \item \textbf{Confidentialité :} Informations sur la confidentialité.
\end{itemize}

\chapter{Menu des outils}

\section{Description}
Vous ouvrez le menu des outils en appuyant sur l'icône \includegraphics[width=0.05\textwidth]{icons/tree.png} (arbre) à gauche de la barre de menu. Un nouveau menu apparaît à gauche avec deux outils :

\begin{figure}[H]
    \centering
    \includegraphics[width=0.25\textwidth]{ToolMenu2.png}
    \caption{Menu des outils.}
\end{figure}

\section{Outils disponibles}
\begin{itemize}
    \item \textbf{Icône GPS :} \includegraphics[width=0.05\textwidth]{icons/gps.png} Indique si le mobile connaît votre position. Si elle passe du gris au rouge, votre position a été trouvée. En appuyant dessus, l'écran se centre sur votre position GPS.
    \item \textbf{Analyse ponctuelle :} Si votre position est visible à l'écran, une icône supplémentaire apparaît. En appuyant dessus, vous obtenez une analyse ponctuelle pour votre position GPS.
\end{itemize}

\begin{figure}[H]
    \centering
    \includegraphics[width=0.25\textwidth]{ToolMenu3.png}
    \caption{Analyse ponctuelle.}
\end{figure}

\section{Recherche de localisation}
Le dernier symbole est une loupe \includegraphics[width=0.05\textwidth]{icons/search.png}. Elle permet de rechercher une localisation en Wallonie. Les résultats proviennent du serveur Nominatim d'OpenStreetMap.

\chapter{Menu des polygones}

\section{Description}
Appuyez sur le symbole \includegraphics[width=0.05\textwidth]{icons/hexagon.png} (hexagone) de la barre de menu pour ouvrir le menu des polygones. La première fois, une boîte apparaît en haut de l'écran avec le texte : \og Tappez ici pour ajouter un polygone \fg.

\begin{figure}[H]
    \centering
    \includegraphics[width=0.25\textwidth]{PolygonMenu.png}
    \caption{Menu des polygones.}
\end{figure}

\section{Procédure : Créer un nouveau polygone}
\begin{instructionbox}{Créer un nouveau polygone}
\begin{enumerate}
    \item Appuyez sur le bouton \og +\fg.
    \item Une boîte de dialogue s'ouvre. Donnez un nom court au nouveau polygone et appuyez sur \og OK \fg.
    \item Une nouvelle carte apparaît en haut de la liste. Elle contient les informations du polygone : le nom, sa surface en mètres carrés, sa circonférence en mètres.
    \item Vous pouvez changer la couleur de la tuile et du polygone sur la carte avec l'icône de couleur en haut à droite de la carte.
    \item Vous pouvez supprimer le polygone en appuyant sur la poubelle \includegraphics[width=0.05\textwidth]{icons/trashcan.png} à gauche de la carte.
    \item Vous pouvez demander une analyse du polygone sur les couches sélectionnées sur nos serveurs. Cela ne fonctionne que si le polygone est inférieur à 250 mètres carrés et bien défini.
    \item Fermez la liste.
    \item En haut, vous voyez maintenant une boîte avec votre polygone sélectionné. Appuyez sur le symbole de centrage pour centrer la vue sur le polygone.
\end{enumerate}
\end{instructionbox}

\begin{figure}[H]
    \centering
    \includegraphics[width=0.25\textwidth]{NewPoly1.png}
    \caption{Nouveau polygone.}
\end{figure}

\section{Procédure : Ajouter des sommets}
\begin{instructionbox}{Ajouter des sommets}
\begin{enumerate}
    \item À gauche, un nouveau menu apparaît avec un \og +\fg. Appuyez dessus pour ajouter des sommets en appuyant sur la carte.
    \item Si vous appuyez sur au moins trois points, un triangle apparaît. L'un des trois côtés du triangle est dessiné avec une ligne épaisse. Cette ligne est divisée en deux avec le prochain sommet que vous placez sur la carte.
    \item Pour changer cette ligne, vous devez appuyer sur un autre sommet.
\end{enumerate}
\end{instructionbox}

\begin{figure}[H]
    \centering
    \includegraphics[width=0.25\textwidth]{NewPoly4.png}
    \caption{Ajout de sommets.}
\end{figure}

\section{Procédure : Supprimer ou déplacer un sommet}
\begin{instructionbox}{Supprimer ou déplacer un sommet}
\begin{enumerate}
    \item Une fois le polygone dessiné, appuyez sur un sommet. Le menu de gauche change : le \og +\fg{} disparaît et un \og -\fg{} et une icône \includegraphics[width=0.05\textwidth]{icons/open_with.png} apparaissent.
    \item Le \og -\fg{} permet de supprimer le sommet sélectionné.
    \item L'icône \includegraphics[width=0.05\textwidth]{icons/open_with.png} permet de déplacer le sommet sélectionné. Après le déplacement, appuyez à nouveau sur l'icône pour arrêter de le déplacer.
    \item Avec un polygone bien défini, vous pouvez ouvrir la liste et appuyer sur l'icône \includegraphics[width=0.05\textwidth]{icons/analytics.png} pour lancer l'analyse.
    \item Les résultats sont présentés par couche. Si vous voyez une petite icône PDF rouge avec le titre de la couche, vous pouvez consulter sa documentation.
\end{enumerate}
\end{instructionbox}

\begin{figure}[H]
    \centering
    \includegraphics[width=0.25\textwidth]{NewPoly5.png}
    \caption{Suppression ou déplacement d'un sommet.}
\end{figure}

\chapter{Sélecteur de couches}

\section{Description}
Appuyez sur l'icône \includegraphics[width=0.05\textwidth]{icons/eye.png} de la barre de menu pour ouvrir le sélecteur de couches.

\section{Contrôle de la visibilité}
Dans la partie supérieure du sélecteur, vous pouvez contrôler la visibilité des marqueurs de localisation et des polygones.

\section{Cartes superposées}
Au milieu, vous observez trois cartes. Chacune peut être une couche de carte. Vous pouvez superposer jusqu'à trois cartes. Pour la couche supérieure, vous pouvez régler une transparence de 50\%.

\section{Accès aux catalogues}
Dans la partie inférieure, vous pouvez accéder aux catalogues :
\begin{itemize}
    \item \textbf{Symboles des couches :} \includegraphics[width=0.05\textwidth]{icons/layers.png} Accédez au catalogue en ligne sur le serveur.
    \item \textbf{Symboles de téléchargement :} \includegraphics[width=0.05\textwidth]{icons/download_for_offline.png} Accédez à un catalogue avec uniquement les cartes que vous avez déjà téléchargées.
\end{itemize}

\section{Procédure : Sélectionner une couche dans un emplacement spécifique}
\begin{instructionbox}{Sélectionner une couche dans un emplacement spécifique}
\begin{enumerate}
    \item Appuyez sur le texte dans la tuile du sélecteur. Si aucune carte n'est sélectionnée, vous pouvez lire \og Appuyez ici pour ajouter une couche du catalogue \fg.
    \item Le menu du catalogue s'ouvre. Si vous appuyez sur l'icône \includegraphics[width=0.05\textwidth]{icons/layers.png}, vous pouvez sélectionner cette couche spécifique. Un cercle vert avec un numéro apparaît à la place de l'icône.
\end{enumerate}
\end{instructionbox}

\section{Procédure : Sélectionner trois couches à la fois}
\begin{instructionbox}{Sélectionner trois couches à la fois}
\begin{enumerate}
    \item Ouvrez le sélecteur de couches.
    \item Appuyez sur le bouton \includegraphics[width=0.05\textwidth]{icons/layers.png} pour ouvrir le catalogue.
    \item Sur la carte de la couche, appuyez sur l'icône des couches pour sélectionner une couche.
    \item Après avoir appuyer, un cercle vert avec un numéro apparaît, indiquant dans quel emplacement de couche se trouve la couche dans le sélecteur.
    \item Vous pouvez sélectionner jusqu'à trois couches. La dernière que vous avez sélectionnée sera toujours à la première position dans le sélecteur, et celle qui dépasse le nombre trois sera désélectionnée.
\end{enumerate}
\end{instructionbox}

\section{Procédure : Retirer une couche d'un emplacement spécifique}
\begin{instructionbox}{Retirer une couche d'un emplacement spécifique}
\begin{enumerate}
    \item Appuyez sur le texte dans la tuile du sélecteur pour ouvrir le catalogue.
    \item Appuyez maintenant sur le cercle vert avec le numéro de l'emplacement pour désélectionner la couche.
\end{enumerate}
\end{instructionbox}

\chapter{Catalogue des couches}

\section{Description}
Dans le catalogue des couches, vous pouvez parcourir toutes les couches disponibles sur nos serveurs.

\begin{figure}[H]
    \centering
    \includegraphics[width=0.25\textwidth]{catalogue.png}
    \caption{Catalogue des couches.}
\end{figure}

\section{Description des catégories}
Nos couches sont divisées par catégories :
\begin{itemize}
    \item Conditions stationnelles
    \item Guide des stations
    \item Cartes de référence
    \item Cartographie des peuplements forestiers
    \item État sanitaire de la pessière
    \item Adéquation des essences aux conditions stationnelles
    \item Essences proposées par les guides des stations
    \item Couches thématiques (catalogue de stations)
\end{itemize}

\section{Description des symboles}
En regardant de près une tuile de couche, vous pouvez voir plusieurs petits symboles à gauche du titre. À droite, vous voyez soit un cercle avec un numéro (lorsque la couche est sélectionnée dans le sélecteur de couches), soit une icône \includegraphics[width=0.05\textwidth]{icons/layers.png}.

Les symboles dans la partie droite sont un aperçu de plusieurs propriétés que vous trouverez en développant davantage la tuile de couche en appuyant dessus :
\begin{itemize}
    \item \includegraphics[width=0.05\textwidth]{icons/download.png} : Indique si vous pouvez télécharger la couche sur votre smartphone pour une utilisation hors ligne.
    \item \includegraphics[width=0.05\textwidth]{icons/disk.png} : Indique que la couche a déjà été téléchargée.
    \item \includegraphics[width=0.05\textwidth]{icons/legend_toggle.png} : Indique s'il y a une légende pour montrer la description par pixel de couleur.
    \item \includegraphics[width=0.05\textwidth]{icons/picture_as_pdfBrown.png} : Indique s'il y a une documentation disponible.
    \item \includegraphics[width=0.05\textwidth]{icons/pentagram.png} : Indique si la couche est sélectionnée pour l'analyse de surface.
    \item \includegraphics[width=0.05\textwidth]{icons/locationPin.png} : Indique si la couche est sélectionnée pour l'analyse ponctuelle.
\end{itemize}

\section{Procédure : Modifier les propriétés d'une couche}
\begin{instructionbox}{Modifier les propriétés d'une couche}
\begin{enumerate}
    \item En appuyant sur une couche, vous voyez la description complète et vous pouvez modifier les propriétés en appuyant dessus.
\end{enumerate}
\end{instructionbox}

\begin{figure}[H]
    \centering
    \includegraphics[width=0.25\textwidth]{catalogueLayerTile.png}
    \caption{Tuile de couche dans le catalogue.}
\end{figure}

\chapter{Analyse}

\section{Analyse ponctuelle}
Forestimator Mobile offre différentes possibilités d'analyse, soit ponctuelle, soit surfacique.

\section{Procédure : Réaliser une analyse ponctuelle}
\begin{instructionbox}{Réaliser une analyse ponctuelle}
\begin{enumerate}
    \item Allez à l'écran principal et appuyez simplement quelque part sur la carte.
    \item Un aperçu court en une ligne apparaît et vous indique la légende et la couleur pour ce point.
    \item Si vous appuyez à nouveau sur cet aperçu, vous réaliserez une analyse avec toutes les couches sélectionnées. En mode en ligne, l'analyse sera effectuée sur nos serveurs et en mode hors ligne, elle sera calculée sur votre smartphone.
    \item Alternativement, vous pouvez appuyer longuement sur la carte pour obtenir directement une analyse complète de toutes les couches sélectionnées.
\end{enumerate}
\end{instructionbox}

\begin{warningbox}{Attention}
Cela ne fonctionne pas avec toutes les cartes sélectionnées. Seule la première carte du sélecteur sera interrogée. De plus, cela ne fonctionnera pas avec toutes les couches du catalogue. Par exemple, la carte IGN ou les couches orthophotographiques ne vous montreront pas d'aperçu.
\end{warningbox}

\begin{figure}[H]
    \centering
    \includegraphics[width=0.25\textwidth]{anaPreview.png}
    \caption{Aperçu d'analyse ponctuelle.}
\end{figure}

\section{Procédure : Réaliser une analyse de surface}
\begin{instructionbox}{Réaliser une analyse de surface}
\begin{enumerate}
    \item Appuyez sur le symbole \includegraphics[width=0.05\textwidth]{icons/hexagon.png} du menu principal.
    \item Dans le menu des polygones, appuyez sur la tuile en haut de l'écran pour accéder à la liste des polygones.
    \item Sélectionnez le polygone que vous souhaitez analyser.
    \item Appuyez maintenant sur l'icône \includegraphics[width=0.05\textwidth]{icons/analytics.png} pour lancer l'analyse.
\end{enumerate}
\end{instructionbox}

\section{Procédure : Sélectionner une couche pour l'analyse}
\begin{instructionbox}{Sélectionner une couche pour l'analyse}
\begin{enumerate}
    \item Ouvrez le sélecteur de couches.
    \item Appuyez ensuite sur l'icône \includegraphics[width=0.05\textwidth]{icons/layers.png} pour ouvrir le catalogue.
    \item Recherchez la couche que vous souhaitez analyser et ouvrez-la en appuyant sur le titre.
    \item Appuyez sur \includegraphics[width=0.05\textwidth]{icons/pentagram.png} pour l'inclure dans l'analyse de surface ou sur \includegraphics[width=0.05\textwidth]{icons/locationPin.png} pour l'inclure dans l'analyse ponctuelle.
    \item Appuyez sur le bouton de fermeture.
    \item Vous pouvez maintenant effectuer votre analyse.
\end{enumerate}
\end{instructionbox}

\begin{figure}[H]
    \centering
    \includegraphics[width=0.25\textwidth]{anaPt.png}
    \caption{Analyse ponctuelle.}
\end{figure}

\chapter{Mode en ligne/hors ligne}

\section{Description}
Pour les personnes loin à l'étranger, il existe un mode hors ligne que vous pouvez utiliser pour consulter les cartes et analyser en déplacement.

\section{Procédure : Passer en mode hors ligne}
\begin{instructionbox}{Passer en mode hors ligne}
\begin{enumerate}
    \item Vous devez avoir au moins une couche téléchargée.
    \item Vous pouvez visualiser les couches téléchargées en appuyant sur l'icône \includegraphics[width=0.05\textwidth]{icons/download_for_offline.png} dans le sélecteur de couches.
    \item Le sélecteur de couches supporte la visualisation d'exactement une carte à la fois en raison des calculs impliqués pour votre smartphone.
    \item Vous pouvez faire des aperçus ainsi que des analyses ponctuelles.
    \item Vous ne pouvez pas faire d'analyse de surface, mais vous pouvez dessiner des polygones.
    \item Vous pouvez utiliser le GPS mais pas rechercher des localisations.
    \item Appuyez sur le bouton avec le texte \og Forestimator en ligne \fg. Vous observerez un changement de couleur et une carte sera chargée en mémoire.
\end{enumerate}
\end{instructionbox}

\section{Procédure : Passer en mode en ligne}
\begin{instructionbox}{Passer en mode en ligne}
\begin{enumerate}
    \item Appuyez sur le bouton avec le texte \og Forestimator terrain \fg{} pour revenir en mode en ligne.
\end{enumerate}
\end{instructionbox}

\begin{warningbox}{Attention}
Si vous avez téléchargé des couches, vous pouvez également mélanger une couche téléchargée avec les autres provenant du serveur. Il y a une petite icône dans le sélecteur qui indique si une couche est disponible en ligne ou si elle provient du disque.
\end{warningbox}

\end{document}

