\documentclass[12pt,a4paper]{report}
\usepackage[utf8]{inputenc}
\usepackage[french]{babel}
\usepackage{graphicx}
\usepackage{hyperref}
\usepackage{float}
\usepackage{caption}
\usepackage{enumitem}
\usepackage{tcolorbox}
\usepackage{geometry}
\usepackage{fancyhdr}
\usepackage{lipsum}
\graphicspath{{./Image/}{/home/jo/app/Forestimator/forestimatorMobile/documentation/manuel/Images/}}
\usepackage[square]{natbib} 
\usepackage{glossaries}

\makeglossaries

\geometry{a4paper, left=25mm, right=25mm, top=25mm, bottom=25mm}

% Configuration des boîtes d'instructions et d'avertissements
\tcbuselibrary{skins,breakable}
\newtcolorbox{instructionbox}[1]{
    colback=white,
    colframe=blue!75!black,
    fonttitle=\bfseries,
    colbacktitle=blue!75!black,
    coltitle=white,
    title=#1,
    breakable,
    enhanced,
    attach boxed title to top left={xshift=5mm, yshift=-3mm},
}

\newtcolorbox{warningbox}[1]{
    colback=white,
    colframe=red!75!black,
    fonttitle=\bfseries,
    colbacktitle=red!75!black,
    coltitle=white,
    title=#1,
    breakable,
    enhanced,
    attach boxed title to top left={xshift=5mm, yshift=-3mm},
}

% Configuration du pied de page
\pagestyle{fancy}
\fancyhf{}
\rfoot{
    \begin{minipage}{0.3\textwidth}
        \centering
        \includegraphics[width=0.15\textwidth]{LogoForestimator.png}
    \end{minipage}
    \begin{minipage}{0.5\textwidth}
        \centering
        \thepage
    \end{minipage}
}

\title{Manuel d'utilisation de l'application Forestimator}
\author{Thierry Thissen et Jonathan Lisein}
\date{\today}

\begin{document}

\newglossaryentry{aptitude}{name={aptitude des essences},
	description={Chaque essence est caractérisée par des niveaux de tolérance particuliers vis-à-vis des facteurs du milieu, ce qui définit son autécologie. Quatre niveaux d’aptitude ont été définis dans le Fichier écologique des essences.
	\textbf{Optimum} : L’essence est parfaitement en adéquation avec la station en termes de vitalité, stabilité et productivité. 
	\textbf{Tolérance} : Certaines caractéristiques de la station engendrent une contrainte pour la vitalité, la stabilité ou la productivité de l’essence. Il y a donc lieu d’adapter la sylviculture à ces contraintes.
	\textbf{Tolérance élargie}: S’il n’est pas envisageable de produire du bois de qualité sur la station, l’essence n’est pas forcément à exclure. Dans ces situations, l’utilisation de l’essence se limite alors à un rôle d’accompagnement pour des raisons écologiques ou sylvicoles.
	\textbf{Exclusion} : L’essence est incapable de se développer à long terme sur la station. Les contraintes environnementales sont rédhibitoires}}
	
\newglossaryentry{guideStation}
{name={guide des stations},
	description={Un guide des stations se présente sous la forme d'un éventail de types de station représentatifs de l'ensemble des contextes forestiers existant au sein d’une région forestière donnée. Il s’appuie sur une typologie des stations établie de manière scientifique, basée sur des critères géomorphologiques et pédologiques relativement pointus, aussi appelée catalogue des stations.}}

\newglossaryentry{FEE}
{name={fichier écologique des essences},
	description={Véritable couteaux suisse de la gestion forestière, cette boite à outils regroupe la description de l'autécologie des essences (fiche-essence) et une grille d'analyse des conditions environnementales (écogramme) qui sont classés en régime hydrique, trophique et climatique.}}

	
\newglossaryentry{polygones}
{
	name={polygone},
	description={se refère à la géométrie permettant de délimiter une surface représentant une parcelle forestière ou une entitée adminisatrive.}
	plural={polygones}
}

\newglossaryentry{online}
{
	name={mode en ligne},
	description={Forestimator Mobile fonctionne avec une connexion internet qui lui permet l'affichage des cartes, la réalisation des recherche de lieux et les analyses ponctuelles et surfacique.}
}

\newglossaryentry{offline}
{
	name={mode hors ligne},
	description={Forestimator Mobile fonctionne sans aucune connexion internet. Ce mode est très utile pour travailler sur le terrain dans des zones non couvertes par le réseaux de télécommunication. Les cartes affichées sont toutes préalablement téléchargées et sauvées sur le smartphone. La réalisation d'analyses ponctuelles est réalisée exclusivement avec ces couches cartographiques sauvées localement.}
}

\newglossaryentry{station}
{
	name={station forestière},
	description={Une étendue de terrain de superficie variable, homogène dans ses conditions
		physiques et biologiques : climat, topographie, sol, composition floristique et structure de la
		végétation spontanée.},
	plural={stations forestières}
}

\newglossaryentry{couche}
{
	name={couche cartographique},
	description={Une information spatialisée couvrant un territoire définis (ex: Wallonie) et stoqué sous format numérique. Par abus de language, une couche cartographique est également désigné par le terme plus courant de \textbf{carte}, même si cette terminologie fait référence à l'impression sous format papier d'une couche cartographique.}
	firstplural={couches cartographiques}
	%firstplural={\glsentrydesc{LED}s (\glsentryplural{LED})}
}


\newglossaryentry{catalogue}
{
	name={catalogue des couches},
	description={L'ensemble des couches cartographiques qui sont classées par thématiques (\textbf{conditions stationnelles}, \textbf{peuplements forestiers}). Il existe une version en ligne du catalogue, ainsi qu'une version hors ligne si certaines couches ont été précédemment téléchargées pour un usage sans connexion internet.}
}

\newglossaryentry{analysePt}
{
	name={analyse ponctuelle},
	description={L'affichage pour une position donnée de la valeur d'une couche cartographique ou d'un ensemble de couches cartographiques, permettant dans ce dernier cas de dresser un bilan global de la situation pour cette position : type de station, composition du peuplement, caractérisation dendrométrique}
}

\newglossaryentry{analyseSurf}
{
	name={analyse surfacique},
	description={Le calcul et l'affichage, pour surface définie au moyen d'un polygone, des valeurs d'une ou de plusieurs couches cartographiques. La proportion de surface pour chaque valeurs est également renseigné, par exemple pour l'aptitude du Hêtre ; 70\% de la surface en Optimum, 30 \% de la surface en Tolérance élargie.}
}

\newglossaryentry{ForestimatorM}
{
	name={Forestimator Mobile},
	description={L'application smartphone ou tablette de Forestimator, distribuée par l'app store pour IOS et par le play store pour Android. Disponible depuis 2025.}
}

\newglossaryentry{ForestimatorWeb}
{
	name={Forestimator Web},
	description={Le portail cartographique web Forestimator, utilisé depuis un ordinateur via un navigateur web, en fonctionnement depuis 2021.}
}

\newglossaryentry{coucheDendro}
{
	name={couches dendrométriques},
	description={L'estimation des dimensions des arbres (diamètre, hauteur, volume, circonférence) et des peuplements pour évaluer leur volume de bois a bénéficié des développements technologiques des outils de télédétection (acquisition aéroportée LiDAR, suivi de la végétation par prise de vue satellitaire multispectrale) et de la puissance des algorithmes d'intelligence artificielle, en particulier ceux fonctionnant au départ d'image. Dans le catalogue de forestimator, une série de couches ont trait aux caractéristiques dendrométriques des peuplements tels qu'ils se présentaient en 2021 : nombre de tige à l'hectare (Nha), surface terrière (Gha), volume de bois à l'hectare (Vha), circonférence et hauteur dominante (Cdom et Hdom) et proportion de feuillus.}
}


\begin{titlepage}
    \centering
    \vspace*{2cm}
    {\Huge \textbf{Manuel d'utilisation de l'application Forestimator}\par}
    \vspace{1cm}
    {\Large Rédaction : Thierry Thissen, Jonathan Lisein.\par}
    \vspace{2cm}
    \includegraphics[width=0.4\textwidth]{LogoUliegeForest.jpg}\par
    \vspace{1cm}
    \includegraphics[width=0.2\textwidth]{LogoForestimator.png}\par
     \vspace{2cm}
     {\Large Logiciel et documentation réalisé au moyen du financement de l'Accord Cadre de Recherches et Vulgarisation Forestières.\par}
    \vfill
    {\large \today\par}
\end{titlepage}

\tableofcontents

\chapter{Manuel d'utilisation de Forestimator Mobile}

\section{Introduction}
Forestimator est un portail cartographique regroupant une série de \glspl{couche}\footnote{en anglais \textit{layers}. Nous désignerons également ces couches par le terme \textbf{carte}, bien que cette terminologie soit à l'origine réservée à la version papier d'une couche cartographique.}  dédiées à la description des forêts et des conditions environnementales en Wallonie. Il permet aux utilisateurs de visualiser des cartes, d'effectuer des analyses sur des surfaces ou simplement pour une position spécifique. Pour plus d'informations, consultez la documentation officielle : \url{https://forestimator.gembloux.ulg.ac.be/documentation/} ainsi que l'article de présentation de Forestimator Web \citep{lisein_forestimator_2022}.

Il existe une version web et une application mobile de Forestimator. La version web s'utilise au bureau via un ordinateur, alors que Forestimator Mobile est conçu aussi bien au bureau que sur le terrain. Ce manuel est dédié à \gls{ForestimatorM}, disponible sur Android et iOS. La première partie de ce document illustre le fonctionnement de Forestimator Mobile. A la fin de la première partie se trouve un glossaire, qui permet de définir le vocabulaire technique utilisé (voir \ref{ref:glossaire}). Les termes définit de manière vulgarisée font référence aux concepts sous-jacents la description cartographique des forêts et à l'interface graphique de Forestimator Mobile. Dans la deuxième partie du document, une mise en pratique plus complète est illustrée au moyen d'exercices.

La description des \glspl{station} et des peuplements forestiers découle de projet parfois très anciens, comme par exemple la carte numérique des sols de Wallonie (sondages entre 1950 et 1990, \citep{bah_legende_2007}), mais sont souvent le résultats de recherches récentes et encore en cours. Par exemple, la cartographie des essences forestières en place par utilisation d'acquisitions de LiDAR aérien combinées à des prises de vues satellitaires fait encore l'objet d'un travail continu en 2026 \citep{bolyn_carte_2020}. Les mises à jours de l'application Forestimator sont donc régulières et permettent d'ajouter les cartes les plus récentes qui sont le fruits des recherches scientifiques de l'unité de Gestion des Ressources Forestières de Gembloux Agro-Bio Tech.


\section{Installation}

\subsection{Avant l'installation}
\begin{itemize}
    \item \gls{ForestimatorM} est disponible pour Android et iOS.
    \item L'application peut être installée en Belgique, aux Pays-Bas, en France, en Allemagne et au Luxembourg.
    \item La plupart des fonctionnalités de Forestimator Mobile requièrent une connexion à internet.
\end{itemize}

\subsection{Procédure d'installation}
\begin{instructionbox}{Procédure d'installation}
\begin{enumerate}
    \item Ouvrez le Google Play Store ou l'App Store d'Apple.
    \item Recherchez le terme \og forestimator \fg.
    \item Appuyez sur \og Installer \fg.
    \item Attendez la fin de l'installation.
    \item Appuyez sur \og Ouvrir \fg{} pour lancer Forestimator pour la première fois.
\end{enumerate}
\end{instructionbox}

\section{Présentation globale}

\subsection{Permissions}
Lors de la toute première utilisation, l'application demandera des permissions. Celles-ci ne sont jamais nécessaires pour le fonctionnement de l'application, mais certaines tâches spécifiques ne peuvent être réalisées sans elles.

\begin{figure}[H]
    \centering
    \includegraphics[width=0.35\textwidth]{Permissions.png}
    \caption{Demande de permission d'utilisation de la localisation GPS de votre smartphone. Votre position sera affichée sur la carte par Forestimator et vous servira de point de repère.}
\end{figure}

\subsection{Écran principal}
Lors du démarrage de l'application, vous verrez l'écran principal avec la carte topographique de l'IGN en arrière-plan. En plus de la carte, Forestimator contient plusieurs menus et boutons interactifs qui sont présentés dans ce manuel.

\begin{figure}[H]
	\centering
	\includegraphics[width=0.35\textwidth]{StartScreen.png}
	\caption{Écran principal de Forestimator.}
\end{figure}

\begin{figure}[H]
    \centering
    \includegraphics[width=0.35\textwidth]{StartScreenDescription.png}
    \caption{Localisation des différents menus et outils accessibles depuis l'écran principal.}
\end{figure}

\subsection{Barre de menu inférieure}
En bas de l'écran, vous trouverez la barre de menu principale qui est constituée de trois éléments :
\begin{itemize}
    \item \textbf{À gauche :} \includegraphics[width=0.05\textwidth]{icons/tree.png} (icône d'arbre) pour ouvrir le menu des outils de géolocation. La géolocalisation est effectuée soit au moyen du GPS de votre smartphone, soit au moyen de recherche d'un lieu via son appellation.
    \item \textbf{Au centre :} \includegraphics[width=0.05\textwidth]{icons/hexagon.png} (icône d'hexagone) pour ouvrir le menu des polygones. Ces \glspl{polygones} représentent des surfaces de sols qui sont associés à une parcelle forestière ou à une parcelle cadastrale.
    \item \textbf{À droite :} \includegraphics[width=0.05\textwidth]{icons/eye.png} (icône d'œil) pour ouvrir le sélecteur de couches qui permet d'accéder au \textbf{\gls{catalogue}}.
\end{itemize}

\subsection{Barre de menu supérieure}
En haut de l'écran, vous trouverez :
\begin{itemize}
    \item \textbf{À gauche :} \includegraphics[width=0.05\textwidth]{icons/settings.png} (icône de paramètres) pour accéder aux réglages globaux et aux informations relatifs à l'application.
    \item \textbf{Au centre :} Dans la bannière se trouve l'intitulé \og Forestimator online \fg{} qui permet de basculer entre les modes en ligne et hors ligne. Le \gls{offline} est une utilisation sans connexion à internet pour un usage dans un territoire non couvert par un réseau mobile\footnote{zones blanches, comme par exemple dans certains vallons forestiers}
\end{itemize}

\begin{warningbox}{Attention}
La carte de l'écran principal de Forestimator peut afficher jusqu'à trois \glspl{couche} simultanément, qui sont surimposées les unes sur les autres. Le niveaux de zoom de certaines couches est restreint, ce qui a pour effet de les faire disparaître si votre niveau de zoom est trop important. Cette situation est renseignée par un signe d'avertissement en haut à droite de l'écran lorsqu'elle se produit.
\end{warningbox}

\subsection{Paramètres}

Dans les paramètres, vous pouvez modifier les permissions accordées et consulter les informations relatives à l'application. Les différentes sections reprises dans les paramètres sont les suivantes :
\begin{itemize}
	\item \textbf{Permissions :} Consultez et modifiez les permissions accordées.
	\item \textbf{À propos :} Indique la version de Forestimator et les informations sur le financement.
	\item \textbf{Contact :} Adresse du site web principal et adresse emails pour poser des questions ou signaler des bugs.
	\item \textbf{Confidentialité :} Informations sur l'utilisation des données à caractère personnel.
\end{itemize}

\begin{figure}[H]
    \centering
    \includegraphics[width=0.35\textwidth]{Settings.png}
    \caption{Menu des paramètres.}
\end{figure}


\section{Catalogue des couches cartographiques}

Le \gls{catalogue} vous permet de parcourir toutes les couches disponibles. Le catalogue sert principalement à sélectionner les couches thématiques à afficher, mais également à consulter la documentation et la légende de chacune des couches, à sélectionner des couches pour les analyses ponctuelles et surfaciques et enfin à effectuer le téléchargement des couches pour leur usage en \gls{offline}. Si certaines couches sont téléchargées, elle seront accessibles via un deuxième catalogue qui regroupe exclusivement les couches sauvées sur le smartphone.

\begin{figure}[H]
	\centering
	\includegraphics[width=0.35\textwidth]{catalogue.png}
	\caption{Catalogue des couches, regroupée par thématique (ici le groupe \textit{conditions stationnelles}).}
\end{figure}

\subsection{Description des catégories}
Les couches cartographiques sont classées par catégories :
\begin{itemize}
	\item \textbf{Cartes de référence} : les couches permettant de se repérer, tels que la carte topographique et les orthophotographies de la Région Wallonne.
	\vspace{0.5cm}
	\item \textbf{Conditions stationnelles} : les couches de descriptions des \glspl{station} en liens avec le \gls{FEE}.
	\vspace{0.5cm}
	\item \textbf{Guide des stations} : d'ores et déjà finalisés pour l'Ardenne \citep{tossens_guide_2024} et la Fagne-Famenne-Calestienne, ces guides proposent une alternative au découpage stationnel du \gls{FEE}.
	\vspace{0.5cm}
	\item \textbf{Cartographie des peuplements forestiers} : description des peuplements en place, notamment pour l'année 2021 durant laquelle une acquisition LiDAR a permis une description fine des peuplements \citep{lejeune_nouvelles_2025}.
	\vspace{0.5cm}
	\item \textbf{État sanitaire de la pessière} : cartographie du dépérissement initié par la crise des scolytes \citep{gilles_spatial_2024}.
	\vspace{0.5cm}
	\item \textbf{Adéquation des essences aux conditions stationnelles} :  carte d'aptitude du \gls{FEE}, une carte par espèce.
	\vspace{0.5cm}
	\item \textbf{Essences proposées par les guides des stations} : carte de vulnérabilité et de recommandation, une carte par espèce.
	%\item Couches thématiques (catalogue de stations)
\end{itemize}

\subsection{Description des symboles associés à une couche cartographique}
En regardant de près l'encart d'une couche dans le catalogue, vous pouvez distinguer plusieurs petits symboles à gauche du nom de celle-ci. À droite se trouve soit un cercle avec un numéro (lorsque la couche est sélectionnée dans le sélecteur de couches), soit une icône \includegraphics[width=0.05\textwidth]{icons/layers.png}.

Les symboles localisés dans la partie droite résument plusieurs caractéristiques qui sont plus amplement détaillées lorsque vous appuyez sur l'encart. Les caractéristiques d'une couche sont structurées de la manière suivante :
\begin{itemize}
	\item \includegraphics[width=0.05\textwidth]{icons/download.png} : Indique si vous pouvez télécharger la couche sur votre smartphone pour une utilisation en \gls{offline}.
	\item \includegraphics[width=0.05\textwidth]{icons/disk.png} : Indique que la couche a déjà été téléchargée.
	\item \includegraphics[width=0.05\textwidth]{icons/legend_toggle.png} : Indique s'il y a une légende pour illustrer la symbologie de la couche, c'est à dire une couleur pour chaque valeur de la carte (ex: rouge pour l'exclusion d'une essence forestière).
	\item \includegraphics[width=0.05\textwidth]{icons/picture_as_pdfBrown.png} : Indique s'il y a une documentation disponible et consultable directement dans Forestimator Mobile.
	\item \includegraphics[width=0.05\textwidth]{icons/pentagram.png} : Indique si la couche est sélectionnée pour l'\gls{analyseSurf}.
	\item \includegraphics[width=0.05\textwidth]{icons/locationPin.png} : Indique si la couche est sélectionnée pour l'\gls{analysePt}.
\end{itemize}

\subsection{Modifier les options d'une couche}
\begin{instructionbox}{Modifier les options d'une couche}
	\begin{enumerate}
		\item En appuyant sur l'encart d'une couche, vous avez accès à la description complète et vous pouvez modifier ses propriétés en appuyant dessus.
	\end{enumerate}
\end{instructionbox}

\begin{figure}[H]
	\centering
	\includegraphics[width=0.35\textwidth]{catalogueLayerTile.png}
	\caption{Encart listant les détails d'une couche cartographique.}
\end{figure}

\section{Sélecteur de couches}

Le sélecteur de couches est l'équivalent du panneau des calques dans les autres logiciels de dessin ou de cartographie (QGIS, illustrator). Il permet d'accéder à toutes les couches du projets, d'en modifier la visibilité et de contrôler leur ordre d'affichage. L'accès au \gls{catalogue} se fait également via le sélecteur de couches.

Appuyez sur l'icône \includegraphics[width=0.05\textwidth]{icons/eye.png} de la barre de menu inférieur pour ouvrir le sélecteur de couche. Le sélecteur de couches est organisé en trois parties. La première en haut permet de contrôler la visibilité des couches de points et de polygones. La deuxième permet l'affichage de un à trois couche(s) thématique(s), et la troisième partie en bas permet un accès direct aux catalogues.

\begin{figure}[H]
	\centering
	\includegraphics[width=0.35\textwidth]{selecteur1.png}
	\caption{Affichage du sélecteur de couches, permettant de contrôler la visibilité des couches contenant des points ou des polygones (en haut), ainsi que les couches thématiques de Forestimator}
\end{figure}

\subsection{Visibilité des couches de points et polygones}
Dans la partie supérieure du sélecteur, vous pouvez contrôler la visibilité des marqueurs de localisation résultant de la recherche de lieu sur base de leur nom, ainsi que la visibilité des polygones délimitant vos parcelles forestières.

\subsection{Sélection des couches thématiques}
Au milieu du sélecteur sont disposés trois encarts permettant de sélectionner autant de couches thématiques. L'affichage de ces trois couches est réalisée de haut en bas, le contenu de la deuxième et troisième couches peut donc être partiellement caché par la première couche. Vous pouvez régler une transparence de 50\% pour cette première couche, de manière à distinguer le contenu des deux autres couches thématique.

\subsection{Accès aux catalogues}
Dans la partie inférieure, vous pouvez accéder aux catalogues :
\begin{itemize}
    \item \textbf{Symboles des couches :} \includegraphics[width=0.05\textwidth]{icons/layers.png} Accédez au catalogue en ligne sur le serveur.
    \item \textbf{Symboles de téléchargement :} \includegraphics[width=0.05\textwidth]{icons/download_for_offline.png} Accédez au catalogue contenant uniquement les cartes que vous avez téléchargées précédemment pour un usage en \gls{offline}. Si aucune couche n'a été téléchargée, ce bouton ne s'affiche pas car ce catalogue est vide.
\end{itemize}

\subsection{Procédures de sélection d'une couche thématique}
\begin{instructionbox}{Sélectionner une couche thématique depuis un des trois encarts du sélecteur}
\begin{enumerate}
    \item Appuyez sur un des trois encarts présent dans le sélecteur de couche. Si aucune carte n'est précédemment sélectionnée, un seul encart est présent dans lequel vous pouvez lire \og Appuyez ici pour ajouter une couche du catalogue \fg.
    \item Le catalogue s'affiche et vous permet de naviguer parmi les différentes couches thématiques. Si vous appuyez sur l'icône \includegraphics[width=0.05\textwidth]{icons/layers.png}, vous pouvez sélectionner cette couche spécifique. Un cercle vert avec un numéro de 1 à 3 apparaît alors à la place de l'icône. Ce numéro représente l'ordre d'affichage de la couche.
\end{enumerate}
\end{instructionbox}

\begin{instructionbox}{Sélectionner trois couches à la fois}
\begin{enumerate}
    \item Ouvrez le sélecteur de couches.
    \item Appuyez sur le bouton \includegraphics[width=0.05\textwidth]{icons/layers.png} pour ouvrir le catalogue online.
    \item Sur l'enacart d'une couche, appuyez sur l'icône \includegraphics[width=0.05\textwidth]{icons/layers.png} pour la sélectionner.
    \item Après avoir appuyé, un cercle vert avec un numéro apparaît, indiquant dans quel emplacement de couche se trouve la couche dans le sélecteur.
    \item Vous pouvez sélectionner jusqu'à trois couches. La dernière que vous avez sélectionnée sera toujours à la première position dans le sélecteur.
\end{enumerate}
\end{instructionbox}

\begin{instructionbox}{Retirer une couche d'un emplacement spécifique du sélecteur}
\begin{enumerate}
    \item Appuyez sur un des encart du sélecteur pour ouvrir le catalogue.
    \item Appuyez maintenant sur le cercle vert avec le numéro représentant la position de la couche pour désélectionner celle-ci.
    \item Appuyez sur le bouton \textit{Fermer} pour cacher le catalogue.
\end{enumerate}
\end{instructionbox}

\section{Menu des outils de géolocalisation}

\begin{figure}[H]
	\centering
	\includegraphics[width=0.35\textwidth]{ToolMenu2B.png}
	\caption{Position des outils de géolocalisation.}
\end{figure}

Vous ouvrez le menu des outils de géolocalisation en appuyant sur l'icône \includegraphics[width=0.05\textwidth]{icons/tree.png} (arbre) à gauche de la barre de menu inférieur. Un nouveau menu apparaît sur la gauche contenant deux ou trois icônes, en fonction du statut de la position GPS de votre smartphone :

\begin{itemize}
	\item \textbf{Icône GPS :} \includegraphics[width=0.05\textwidth]{icons/gps.png} Indique si votre téléphone mobile a accès à votre position. Si elle passe du gris au rouge, votre position a été déterminée. En appuyant dessus, l'écran se centre sur votre position GPS.
	\item \textbf{Analyse ponctuelle :} Si votre position est visible à l'écran, une icône supplémentaire apparaît. En appuyant dessus, vous obtenez une \textbf{analyse ponctuelle} pour votre position GPS. L'analyse ponctuelle est une description, pour une position donnée, du contenu de l'ensemble des couches cartographiques les plus intéressantes.
	\item Le dernier symbole est une loupe \includegraphics[width=0.05\textwidth]{icons/search.png}. Elle permet de \textbf{rechercher une localisation} en Wallonie sur base d'un nom de lieu ou d'un mot clé. % Les résultats proviennent du serveur Nominatim d'OpenStreetMap.
\end{itemize}

\begin{figure}[H]
	\centering
	\includegraphics[width=0.35\textwidth]{ToolMenu3.png}
	\caption{Icônes du menu de géolocalisation. De haut en bas : 1) analyse pour la position GPS actuelle du smartphone, 2) centrage de la carte sur la position GPS et 3) outil de recherche d'un lieu sur base de son nom}
\end{figure}

\begin{figure}[H]
	\centering
	\includegraphics[width=0.35\textwidth]{Recherche1.png}
	\caption{Résultat de recherche d'un lieu sur base de son nom, exemple avec Gembloux}
\end{figure}


\section{Menu des polygones}


Appuyez sur le symbole \includegraphics[width=0.05\textwidth]{icons/hexagon.png} (hexagone) de la barre de menu pour ouvrir le menu des \glspl{polygones}. La première fois, un encart apparaît en haut de l'écran avec le texte : \og Tappez ici pour ajouter un polygone \fg. Les polygones permettent de délimiter une surface de manière à la visualiser sur la carte de Forestimator et a y effectuer des analyses, par exemple dans le but de décrire les peuplements en place. Les polygones sont dessinés en ajoutant progressivement les points délimitant son périmètre. Ces points se nomment \textit{vertex} en anglais ou \textit{sommet} en français. Outre la géométrie des polygones, vous avez la possibilité de choisir un nom ainsi qu'une couleur de rendu.

\begin{figure}[H]
	\centering
	\includegraphics[width=0.35\textwidth]{PolygonMenu.png}
	\caption{Menu des polygones.}
\end{figure}

\subsection{Procédure de création de polygone, d'ajout et d'édition de sommet}
\begin{instructionbox}{Créer un nouveau polygone}
	\begin{enumerate}
		\item Appuyez sur le bouton \og +\fg.
		\item Une boîte de dialogue s'ouvre. Donnez un nom court au nouveau polygone et appuyez sur \og OK \fg.
		\item Un nouvel élément apparaît en haut de la liste des couches cartographiques de polygone. Les informations relative au polygone y sont affichées : le nom, sa surface et sa circonférence.
		\item Vous pouvez changer la couleur du polygone tel qu'il est représenté sur la carte au moyen de l'icône de couleur en haut à droite.
		\item Vous pouvez supprimer le polygone en appuyant sur l'icone de corbeille. \includegraphics[width=0.05\textwidth]{icons/trashcan.png} à gauche.
		\item Vous pouvez effectuer une analyse surfacique pour ce polygone. Cette analyse surfacique requiert une certaine puissance de calcul et est dès lors effectué sur un serveur de Gembloux Agro-Bio tech. Une connexion à internet est donc obligatoirement nécessaire pour cette opération.
		\item Fermez la liste.
		\item En haut, vous voyez maintenant un encart portant le nom de votre polygone. Appuyez sur le symbole de centrage pour centrer la vue sur le polygone.
	\end{enumerate}
\end{instructionbox}

\begin{figure}[H]
	\centering
	\includegraphics[width=0.35\textwidth]{NewPoly1.png}
	\caption{Nouveau polygone créé par l'utilisateur.}
\end{figure}

\begin{instructionbox}{Ajouter des sommets/vertex}
	\begin{enumerate}
		\item À gauche, un nouveau menu apparaît avec un \includegraphics[width=0.05\textwidth]{icons/add_circle.png}. Appuyez dessus pour ajouter des sommets en appuyant sur la carte.
		\item Si vous appuyez sur au moins trois points, un triangle apparaît. L'un des trois côtés du triangle est représenté au moyen d'une ligne plus épaisse. C'est ce segment de droite qui sera divisée en deux par le prochain sommet que vous placerez sur la carte.
		\item Pour changer le segment de droite à modifier, vous pouver appuyer sur un autre sommet.
	\end{enumerate}
\end{instructionbox}

\begin{figure}[H]
	\centering
	\includegraphics[width=0.35\textwidth]{NewPoly4.png}
	\caption{Ajout de sommets.}
\end{figure}

\begin{instructionbox}{Supprimer ou déplacer un sommet}
	\begin{enumerate}
		\item Une fois le polygone dessiné, appuyez sur un sommet. Le menu de gauche change : le \includegraphics[width=0.05\textwidth]{icons/add_circle.png} disparaît et un \includegraphics[width=0.05\textwidth]{icons/remove_circle.png} et une icône \includegraphics[width=0.05\textwidth]{icons/open_with.png} apparaissent.
		\item Le \includegraphics[width=0.05\textwidth]{icons/remove_circle.png} permet de supprimer le sommet sélectionné.
		\item L'icône \includegraphics[width=0.05\textwidth]{icons/open_with.png} permet de déplacer le sommet sélectionné. Après le déplacement, appuyez à nouveau sur l'icône de manière à finaliser l'édition.
	\end{enumerate}
\end{instructionbox}

\begin{figure}[H]
	\centering
	\includegraphics[width=0.35\textwidth]{NewPoly5.png}
	\caption{Suppression ou déplacement d'un sommet.}
\end{figure}

\section{Analyses}

Forestimator Mobile offre deux possibilités d'analyse, soit ponctuelle, soit surfacique.

\subsection{Analyse ponctuelle}

L'\gls{analysePt} est très simple à réaliser et permet de questionner la valeur que prends une ou plusieurs couches thématiques pour une position donnée.
\begin{instructionbox}{Analyse ponctuelle en une étape}
	\begin{enumerate}
		\item Depuis l'écran principal de Forestimator, effectuer un \textbf{appui long} sur la carte pour obtenir directement une analyse complète de toutes les couches sélectionnées.
	\end{enumerate}
\end{instructionbox}

\begin{figure}[H]
	\centering
	\includegraphics[width=0.35\textwidth]{anaPt.png}
	\caption{Analyse ponctuelle : un \textbf{appui long} sur l'écran principal de Forestimator permet de visualiser les valeurs, pour cette position, d'un ensemble de couches cartographiques. }
\end{figure}

\begin{instructionbox}{Analyse ponctuelle en deux étapes}
	\begin{warningbox}{Attention}
		L'analyse ponctuelle en deux étapes ne fonctionne que si la première couches sélectionnées n'est pas un fond de carte (carte topo ou orthophotomosaique).
	\end{warningbox}
\begin{enumerate}
	\item Dans le sélectionneur de couche, veillez à ce que la première couche ne soit pas un fond de carte. Utiliser par exemple la couche du guide des stations.
    \item Depuis l'écran principal de Forestimator, appuyez simplement quelque part sur la carte (\textbf{appui court}).
    \item Un aperçu court en une ligne apparaît dans une fenêtre et vous renseigne, pour cette position, la légende et la couleur de la couche cartographique numéro 1.
    \item Si vous appuyez à nouveau sur cette fenêtre d'aperçu, vous réaliserez une analyse avec une série de couches sélectionnées.
\end{enumerate}
\end{instructionbox}

\begin{figure}[H]
    \centering
    \includegraphics[width=0.35\textwidth]{anaPreview.png}
    \caption{Analyse ponctuelle : un \textbf{appui court} sur une carte thématique permet d'afficher la légende pour cette position, dans cet exemple la station est un \textit{versant chaud sur sol superficiel}. Un deuxième appui court sur la légende pour cette position permet la visualisation des valeurs, toujours pour cette position, d'un ensemble de couches cartographiques. }
\end{figure}

\subsection{Tableau d'aptitude et de recommandation des essences}

Les résultats de l'analyse ponctuel sont présenté couche par couche, à l'exception des aptitudes des essences du \gls{FEE} et des recommandations du \gls{guideStation} qui sont présentés sous forme de tableau. En effet, un tableau d'aptitude liste les essences pour chacunes des trois classes d'aptitude suivante : \textbf{Optimum}, \textbf{Tolérance} et \textbf{Tolérance élargie}. De manière similaire, un tableau regroupe les recommandations issues du guide des stations (figure \ref{fig:tabApt}).
	
\begin{figure}[H]
	\ref{fig:tabApt}
		\centering
		\includegraphics[width=0.7\textwidth]{tabAptetRec.png}
		\caption{Analyse ponctuelle : Les aptitudes des essences du \gls{FEE} (gauche) ainsi que les recommandations issues du \gls{guideStation} (droite) pour une position donnée sont regroupé sous forme de tableaux. }
\end{figure}
	

\subsection{Analyse surfacique}
\begin{instructionbox}{Réaliser une analyse de surface}
\begin{enumerate}
    \item Appuyez sur le symbole \includegraphics[width=0.05\textwidth]{icons/hexagon.png} du menu principal (en bas de la carte).
    \item Accédez au le menu des \glspl{polygones} et appuyant sur l'encart en haut de l'écran.
    \item Sélectionnez le polygone pour lequel vous souhaitez effectuer l'analyse surfacique.
    \item Appuyez maintenant sur l'icône \includegraphics[width=0.05\textwidth]{icons/analytics.png} pour lancer l'analyse.
    \item Les résultats de l'analyse sont présentés par couche.
\end{enumerate}
\end{instructionbox}

\begin{figure}[H]
	\centering
	\includegraphics[width=0.35\textwidth]{anaSurf1.png}
	\caption{Résultat d'une analyse surfacique pour un polygone.}
\end{figure}

\subsection{Procédure : Sélectionner une couche thématique pour l'analyse}

Les analyses ponctuelles et surfacique sont des analyses effectuées sur une série de couche cartographique. Cette série est prédéfinie, mais vous pouvez ajouter d'autre couche de votre intérêt de manière à les utiliser lors de vos analyses.

\begin{instructionbox}{Sélectionner une couche pour l'analyse}
\begin{enumerate}
    \item Ouvrez le sélecteur de couches.
    \item Appuyez ensuite sur l'icône \includegraphics[width=0.05\textwidth]{icons/layers.png} pour ouvrir le catalogue.
    \item Recherchez la couche que vous souhaitez analyser et ouvrez-la en appuyant sur le titre.
    \item Appuyez sur \includegraphics[width=0.05\textwidth]{icons/pentagram.png} pour l'inclure dans l'analyse de surface ou sur \includegraphics[width=0.05\textwidth]{icons/locationPin.png} pour l'inclure dans l'analyse ponctuelle.
    \item Appuyez sur le bouton de fermeture.
    \item Vous pouvez maintenant effectuer votre analyse.
\end{enumerate}
\end{instructionbox}

\begin{figure}[H]
    \centering
    \includegraphics[width=0.35\textwidth]{anaPt.png}
    \caption{Analyse ponctuelle.}
\end{figure}


\section{Références, bibliographie et glossaire }
%\renewcommand{\refname}{}
\subsection{Documentation sous forme de capsule vidéo}

\subsubsection{Forestimator, l'application qui cartographie les forêts wallonnes}

Le reportage vidéo de Julie Fohal réalisé début 2025 présente l'application forestimator et son usage sur le terrain. Le reportage est disponible sous le lien suivant : \url{https://www.tvlux.be/actu/info/forestimator-l-application-qui-cartographie-les-forets-wallonnes_48278}.

\begin{figure}[H]
	\centering
	\includegraphics[width=0.35\textwidth]{reportageJulieFohal.jpg}
	\caption{Reportage sur Forestimator Mobile en mars 2025 contenant une interview du Professeur Philippe Lejeune et de Jonathan Lisein.}
\end{figure}


\bibliography{forestimator.bib}
\bibliographystyle{plainnatGL_v2}


\glsaddall

\printglossary[nonumberlist]\label{ref:glossaire}

\chapter{Exercices pratiques}

\section{Analyse ponctuelle}

Cet exercice illustre pas à pas les étapes permettant d'afficher une carte des hauteurs des arbres, d'effectuer une analyse ponctuelle de manière à déterminer la hauteur dominante et d'exporter le résultat de l'analyse sous forme d'un rapport au format pdf. 
\subsection{naviger jusqu'au bois de Grand-Leez}

La navigation peut s'effectuer de manière manuelle, c'est à dire en dézoomant la carte principale, en navigant vers le bois de Grand-Leez qui se situe à l'ouest de Gembloux, puis en zoomant à nouveau. Mais il est plus simple d'utiliser l'outil de géolocalisation et d'entrer le nom du village (voir figure \ref{fig:geolocGL})

\begin{figure}[H]
	\label{fig:geolocGL}
	\centering
	\includegraphics[width=0.8\textwidth]{navGrandLeez.png}
	\caption{Navigation jusqu'à Grand-Leez au moyen de la recherche de lieu par nom.}
\end{figure}

\section{Utilisation du mode hors ligne}

\subsection{Mode en ligne/hors ligne}


%Pour les personnes loin à l'étranger, il existe un mode hors ligne que vous pouvez utiliser pour consulter les cartes et analyser en déplacement.

\subsection{Avant d'utiliser le mode hors ligne}
\begin{itemize}
	\item Vous devez avoir au moins une couche téléchargée.
	\item Vous pouvez visualiser les couches téléchargées en appuyant sur l'icône \includegraphics[width=0.05\textwidth]{icons/download_for_offline.png} dans le sélecteur de couches.
	\item En \gls{offline}, vous ne  pourrez visualiser qu'une seule et unique couche cartographique à la fois. Une seule couche sera donc affichée dans le sélecteur de couches, de manière à ne pas solliciter trop de puissance de lecture et de calcul sur votre smartphone. 
	\item Vous pouvez réaliser des analyses ponctuelles.
	\item Vous ne pouvez pas faire d'\glspl{analyseSurf}.
	\item Vous pouvez utiliser votre propre GPS pour vous géolocaliser, mais vous ne pourrez pas utiliser l'outil de recherche des localisations sur base de leur nom.
\end{itemize}


\begin{instructionbox}{Passer en mode hors ligne}
	\begin{enumerate}
		\item Appuyez sur le bouton avec le texte \og Forestimator online \fg. La bannière de forestimator passera du bleu au vert et une de vos couches préalablement enregistrées sera affichée à l'écran.
	\end{enumerate}
\end{instructionbox}

\begin{figure}[H]
	\centering
	\includegraphics[width=0.35\textwidth]{offlineMode.png}
	\caption{Mode hors ligne.}
\end{figure}


\begin{instructionbox}{Passer en mode en ligne}
	\begin{enumerate}
		\item Appuyez sur le bouton avec le texte \og Forestimator terrain \fg{} pour revenir en mode en ligne.
	\end{enumerate}
\end{instructionbox}

\begin{warningbox}{Attention}
	Si vous avez téléchargé des couches, vous pouvez également mélanger une couche téléchargée avec les autres provenant du serveur. Il y a une petite icône dans le sélecteur qui indique si une couche est disponible en ligne ou si elle provient du disque.
\end{warningbox}

%-exo : télécharger une carte pour usage hors ligne : guide des stations et zbio


\end{document}
